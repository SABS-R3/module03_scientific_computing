\documentclass[a4paper]{article}

\usepackage{enumitem}

\usepackage{xltxtra}
\usepackage{polyglossia}
\usepackage{fancyhdr}
\usepackage{geometry}
\usepackage{enumitem}
\usepackage{versions}
\usepackage{hyperref}

%\includeversion{solution}
\excludeversion{solution}

\geometry{a4paper,left=15mm,right=15mm,top=20mm,bottom=20mm}
\pagestyle{fancy}
\chead{UNIQ+ Summer School - Scientific Python}
\rhead{\today}
\cfoot{\thepage}

\setlength{\headheight}{23pt}
\setlength{\parindent}{0.0in}
\setlength{\parskip}{0.0in}


\usepackage{listings} %iclude code in your document
\usepackage{color}
\definecolor{deepblue}{rgb}{0,0,0.5}
\definecolor{deepred}{rgb}{0.6,0,0}
\definecolor{deepgreen}{rgb}{0,0.5,0}

\lstloadlanguages{Matlab} %use listings with Matlab for Pseudocode
\lstnewenvironment{pseudocode}[1][]
{\lstset{
  language=Matlab,
  basicstyle=\scriptsize, 
  keywordstyle=\color{blue},
  xleftmargin=.04\textwidth,#1}}
{}
\lstnewenvironment{python}[1][]
{\lstset{
  language=Python,
  basicstyle=\scriptsize, 
  keywordstyle=\color{deepblue},
  otherkeywords={self},             % Add keywords here
  keywordstyle=\color{deepblue},
  emph={MyClass,__init__,__str__},          % Custom highlighting
  emphstyle=\color{deepred},    % Custom highlighting style
  stringstyle=\color{deepgreen},
  xleftmargin=.04\textwidth,#1}}
{}
\newcommand{\cf}[1]{\lstinline
[
  language=Python,
  keywordstyle=\color{deepblue},
  otherkeywords={self},             % Add keywords here
  keywordstyle=\color{deepblue},
  emph={MyClass,__init__,__str__},          % Custom highlighting
  emphstyle=\color{deepred},    % Custom highlighting style
  stringstyle=\color{deepgreen}
]
{#1}}



\begin{document}

\section*{Exercises: Optimisation and Inverse Modelling}

\vspace{0,75cm}


\subsection*{Exercise 1 - Calculating derivatives}

\begin{enumerate}[label=\Alph*]
\item Implement the 2D \href{https://en.wikipedia.org/wiki/Rosenbrock_function}{Rosenbrock
function} both as a normal Python
function, and as a mathematical expression using Sympy.

\item Calculate the numerical derivative of the rosenbrock function using a central
  finite difference scheme and compare with the symbolic result using `sympy.diff`. Plot
    the error versus step size $h$ and check if it scales by the expected rate of
    $\mathcal{O}(h^2)$. If not, why?

\item Manually write down (in Python code) the required operations to calculate the
    derivative of the Rosenbrock function using forward mode automatic differentiation.
    Check that the calculated derivative is correct to working precision. Do the same
    for reverse mode automatic differentiation. 
\end{enumerate}

\subsection*{Exercise 2 - Optimisation}


\begin{enumerate}[label=\Alph*]
  \item Implement a simple 2D quadratic function $f(x, y) = x^2 + y^2$, the 2D \href{https://en.wikipedia.org/wiki/Rosenbrock_function}{Rosenbrock
function}, and the 2D \href{https://en.wikipedia.org/wiki/Rastrigin_function}{Rastrigin
    function}, and visualise these
    functions over the domain $-5 < x < 5$ and $-5 < y < 5$
  \item Use [Scipy](https://docs.scipy.org/doc/scipy/reference/optimize.html) to
    minimise each of these functions, starting at $x = y = 2.5$, using the following methods:
        \begin{itemize}
        \item Nelder-Mead Simplex
        \item Conjugate Gradient
        \item BFGS Quasi-Newton
        \item Newton-CG
        \item SHG Global Optimisation
        \end{itemize}
        In each case perform the optimisation with and without a jacobian. To calculate
        the jacobian use automatic differentiation via the \href{https://github.com/HIPS/autograd}{autograd} library.
        
    \item Plot the results for the previous question by plotting the parameter values at
      each iteration of the algorithm overlayed on the visualisation of the function to
      be minimised, and label the figure with the number of iterations and function/jacobian/hessian
      evaluations. What can you say about the performance of each algorithm for each
      function? What are the characteristics of each function that make it
      difficult/easy for the optimisation method?
\end{enumerate}
      

\subsection*{Exercise 3 - Model Fitting}

\begin{enumerate}[label=\Alph*]
  \item Generate fake data points by evaluating the function $f(x) = -2*x**3 +
    3*x**2 + x + 1$ over $n=100$ points $x_i$ evenly spaced between $0 \le x \le 2$. Add
    normally distributed noise to each function evaluation so that each data point $y_i$
    is (independently) sampled using $y_i \sim f(x_i) + \mathcal{N}(0, 1)$.

  \item Implement a polynomial regression routine using an given number of regressors
    $m$, and use it to fit to the fake data. Plot the fitted function so that you can
    evaluate the fit by eye. Then calculate the residual, and the average
    Leave-one-out cross validation error. Plot the residual and cross validation versus
    $m$ to determine the ideal number of regressors $m$.

\end{enumerate}
    

\subsection*{Exercise 4 - ODE Model Fitting}

\begin{enumerate}[label=\Alph*]
    \item The logistic growth equation is given by

      $$ \frac{dy}{dt} = r y (1 - y/k) $$

      Solve this equation using $r = 10$, $k = 1$, and intial condition of $y(0) = p_0
      = 1e-2$ using `scipy.integrate.odeint`, and compare with the exact solution
      given by

      $$ y(t) = \frac{k}{1 + (k/p_0 - 1) \exp(-r t)} $$

  \item If $\mathbf{\omega} = [r, k] $ is the parameters and $f(\omega, y)$ is the RHS
    of the ODE, calculate $\partial f/ \partial \omega$ and $\partial f/ \partial y$,
    and use forward sensitivity analysis to calculate the sensitivities $\partial y(t) /
    \partial r$ and $\partial y(t) / \partial k$. Compare this to the exact solution for
    the sensitivities (i.e. calculate the derivative of the the exact solution above)

  \item Generate some fake data using $r = 10$, $k = 1$, and $p_0 = 1e-2$, using
    independent normally distributed measurement noise with a variance of 0.1.  Attempt
    to recover the true parameters $r,k$ using an initial guess of $r=1$, $k=5$ using
    the Nelder-Mead, BFGS, and Newton-CG optimisation methods. 

\end{enumerate}

\end{document}
