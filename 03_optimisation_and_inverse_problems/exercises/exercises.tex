\documentclass[a4paper]{article}

\usepackage{xltxtra}
\usepackage{polyglossia}
\usepackage{fancyhdr}
\usepackage{geometry}
\usepackage{enumitem}
\usepackage{versions}
\usepackage{hyperref}

%\includeversion{solution}
\excludeversion{solution}

\geometry{a4paper,left=15mm,right=15mm,top=20mm,bottom=20mm}
\pagestyle{fancy}
\chead{UNIQ+ Summer School - Scientific Python}
\rhead{\today}
\cfoot{\thepage}

\setlength{\headheight}{23pt}
\setlength{\parindent}{0.0in}
\setlength{\parskip}{0.0in}


\usepackage{listings} %iclude code in your document
\usepackage{color}
\definecolor{deepblue}{rgb}{0,0,0.5}
\definecolor{deepred}{rgb}{0.6,0,0}
\definecolor{deepgreen}{rgb}{0,0.5,0}

\lstloadlanguages{Matlab} %use listings with Matlab for Pseudocode
\lstnewenvironment{pseudocode}[1][]
{\lstset{
  language=Matlab,
  basicstyle=\scriptsize, 
  keywordstyle=\color{blue},
  xleftmargin=.04\textwidth,#1}}
{}
\lstnewenvironment{python}[1][]
{\lstset{
  language=Python,
  basicstyle=\scriptsize, 
  keywordstyle=\color{deepblue},
  otherkeywords={self},             % Add keywords here
  keywordstyle=\color{deepblue},
  emph={MyClass,__init__,__str__},          % Custom highlighting
  emphstyle=\color{deepred},    % Custom highlighting style
  stringstyle=\color{deepgreen},
  xleftmargin=.04\textwidth,#1}}
{}
\newcommand{\cf}[1]{\lstinline
[
  language=Python,
  keywordstyle=\color{deepblue},
  otherkeywords={self},             % Add keywords here
  keywordstyle=\color{deepblue},
  emph={MyClass,__init__,__str__},          % Custom highlighting
  emphstyle=\color{deepred},    % Custom highlighting style
  stringstyle=\color{deepgreen}
]
{#1}}



\begin{document}

\section*{Exercises: Optimisation and Inverse Modelling}

\vspace{0,75cm}


\subsection*{Exercise 1 - Calculating derivatives}

(a) write rosenbrock function in sympy
(b) code up numerical differentiation of a function (rosenbrock) and compare with
symbolic result (using sympy.diff)
(c) use forward and reverse mode autodiff (library or get them to code up?), vary the
number of inputs and time it, along with numerical and symbolic diff

Aim: students have an understanding of symbolic, numerical, and auto diff and the
comparitive accuracies and time-to-evaluate

\subsection*{Exercise 2 - Optimisation}

(a) Code up a couple of simple optimisation algs in 1D:
  - bisection 
  - gradient descent
  - stochastic gradient descent
  - newtons-rhapson
  - test and visualise the operation on x**2 (convex) and x**2 + np.exp(-5*(x - .5)**2
  (non-convex functions)
  - compare against library functions (take one step at a time)

(b) Use a range of library optimisation algs from these classes of methodso:
  - Simplex (Nelder–Mead)
  - Gradient (Conjugate Gradient, L-BFGS-B)
  - Quasi-Newton (BFGS)
  - Newton (Newton-CG)
  - Global optimisation (CMA-ES)

Apply to the following (2D) problems
  a) an easy convex function
  b) non-convex (Rosenbrock function?)
  c) multi-modal (bunch of gaussians?)

Aim: Get the students to understand the differences between the main classes of
optimisation algorithms and their performance on different types of functions


\subsection*{Exercise 3 - Model Fitting}

- Code up a polynomial regression routine for an arbitrary function, takes in the number
of regressors (n)
- Use on a noisy dataset
  - Visualise fitting results versus n
  - evaluate model performance using residuals (bad) + Leave-one-out cross validation
  (good)
- Improve the result for high n by ridge regression

Aim: Students are aware of the dangers of over-fitting complex models and know about
techniques to reduce this (LOOCV, regularisation or priors)


\subsection*{Exercise 4 - ODE Model Fitting}

if you have time:
  (a) code up pure-python ode integrator. code up forward and adjoint sensitivity calculating using ode integrator, compare
      with forward and reverse-mode automatic differentiation (use logistic model as the ode)
else:
  (a) use scipy odeint and autograd to do the same

(b) fit an ode model (logistic) to noisy "data" using CMA-ES, CG and Quasi-Newton

Aim: students understand how to calculate sensitivities of and ode model and use these
for model fitting

\begin{solution}
\begin{python}
pi = 3.14159265358979312

my_pi = 1.

for i in range(1, 100000):
    my_pi *= 4 * i ** 2 / (4 * i ** 2 - 1.)

my_pi *= 2

print(pi)
print(my_pi)
print(abs(pi - my_pi))
\end{python}
\end{solution}


\end{document}
