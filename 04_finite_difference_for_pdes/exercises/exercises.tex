\documentclass[a4paper]{article}

\usepackage{xltxtra}
\usepackage{polyglossia}
\usepackage{fancyhdr}
\usepackage{geometry}
\usepackage{enumitem}
\usepackage{versions}
\usepackage{hyperref}

%\includeversion{solution}
\excludeversion{solution}

\geometry{a4paper,left=15mm,right=15mm,top=20mm,bottom=20mm}
\pagestyle{fancy}
\chead{UNIQ+ Summer School - Scientific Python}
\rhead{\today}
\cfoot{\thepage}

\setlength{\headheight}{23pt}
\setlength{\parindent}{0.0in}
\setlength{\parskip}{0.0in}


\usepackage{listings} %iclude code in your document
\usepackage{color}
\definecolor{deepblue}{rgb}{0,0,0.5}
\definecolor{deepred}{rgb}{0.6,0,0}
\definecolor{deepgreen}{rgb}{0,0.5,0}

\lstloadlanguages{Matlab} %use listings with Matlab for Pseudocode
\lstnewenvironment{pseudocode}[1][]
{\lstset{
  language=Matlab,
  basicstyle=\scriptsize, 
  keywordstyle=\color{blue},
  xleftmargin=.04\textwidth,#1}}
{}
\lstnewenvironment{python}[1][]
{\lstset{
  language=Python,
  basicstyle=\scriptsize, 
  keywordstyle=\color{deepblue},
  otherkeywords={self},             % Add keywords here
  keywordstyle=\color{deepblue},
  emph={MyClass,__init__,__str__},          % Custom highlighting
  emphstyle=\color{deepred},    % Custom highlighting style
  stringstyle=\color{deepgreen},
  xleftmargin=.04\textwidth,#1}}
{}
\newcommand{\cf}[1]{\lstinline
[
  language=Python,
  keywordstyle=\color{deepblue},
  otherkeywords={self},             % Add keywords here
  keywordstyle=\color{deepblue},
  emph={MyClass,__init__,__str__},          % Custom highlighting
  emphstyle=\color{deepred},    % Custom highlighting style
  stringstyle=\color{deepgreen}
]
{#1}}



\begin{document}

\section*{Exercises: Finite Differences and PDEs}

\vspace{0,75cm}


\subsection*{Exercise 1 - Laplace equation}

- construct matrix for laplace and solve
- compare dense versus sparse representation

\subsection*{Exercise 1 - time integration}

- heat eq (explicit time)
- biharmonic eq (implicit time)
- kuramoto-sivashinsky (combination - IMEX)
- method of lines - scipy odeint - using autodiff for derivatives

\subsection*{Exercise 1 - Perona–Malik Image denoising}

- load image
- try regular diffusion to reduce noise
- discretise and solve perona-malik equation to preserve edges

 
\begin{solution}
\begin{python}
pi = 3.14159265358979312

my_pi = 1.

for i in range(1, 100000):
    my_pi *= 4 * i ** 2 / (4 * i ** 2 - 1.)

my_pi *= 2

print(pi)
print(my_pi)
print(abs(pi - my_pi))
\end{python}
\end{solution}


\end{document}
