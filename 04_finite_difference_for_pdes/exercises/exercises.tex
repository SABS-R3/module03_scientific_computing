\documentclass[a4paper]{article}

\usepackage{enumitem}
\usepackage{amsmath}

\usepackage{xltxtra}
\usepackage{polyglossia}
\usepackage{fancyhdr}
\usepackage{geometry}
\usepackage{enumitem}
\usepackage{versions}
\usepackage{hyperref}

%\includeversion{solution}
\excludeversion{solution}

\geometry{a4paper,left=15mm,right=15mm,top=20mm,bottom=20mm}
\pagestyle{fancy}
\chead{UNIQ+ Summer School - Scientific Python}
\rhead{\today}
\cfoot{\thepage}

\setlength{\headheight}{23pt}
\setlength{\parindent}{0.0in}
\setlength{\parskip}{0.0in}


\usepackage{listings} %iclude code in your document
\usepackage{color}
\definecolor{deepblue}{rgb}{0,0,0.5}
\definecolor{deepred}{rgb}{0.6,0,0}
\definecolor{deepgreen}{rgb}{0,0.5,0}

\lstloadlanguages{Matlab} %use listings with Matlab for Pseudocode
\lstnewenvironment{pseudocode}[1][]
{\lstset{
  language=Matlab,
  basicstyle=\scriptsize, 
  keywordstyle=\color{blue},
  xleftmargin=.04\textwidth,#1}}
{}
\lstnewenvironment{python}[1][]
{\lstset{
  language=Python,
  basicstyle=\scriptsize, 
  keywordstyle=\color{deepblue},
  otherkeywords={self},             % Add keywords here
  keywordstyle=\color{deepblue},
  emph={MyClass,__init__,__str__},          % Custom highlighting
  emphstyle=\color{deepred},    % Custom highlighting style
  stringstyle=\color{deepgreen},
  xleftmargin=.04\textwidth,#1}}
{}
\newcommand{\cf}[1]{\lstinline
[
  language=Python,
  keywordstyle=\color{deepblue},
  otherkeywords={self},             % Add keywords here
  keywordstyle=\color{deepblue},
  emph={MyClass,__init__,__str__},          % Custom highlighting
  emphstyle=\color{deepred},    % Custom highlighting style
  stringstyle=\color{deepgreen}
]
{#1}}



\begin{document}

\section*{Exercises: Finite Differences and PDEs}

\vspace{0,75cm}

\subsection*{Exercise 1 - Laplace equation}

\begin{enumerate}[label=\alph*.]
\item Solve the following Poisson equation on the 1D domain $0 \le x \le 1$, using
  finite differences

  $$-\frac{\partial^2 u}{\partial x^2} = 1$$ 
  $$u(0) = 0$$ 
  $$u(1) = 0$$ 

  Derive the analytical solution and compare against your numerical one.

\item Solve the same Poisson equation, this time using a Neumann boundary condition

  $$-\frac{\partial^2 u}{\partial x^2} = 1$$ 
  $$u(0) = 0$$ 
  $$\left. \frac{\partial u}{\partial x} \right|_{x=1} = 0$$ 

  Derive the analytical solution and compare against your numerical one.

\item Solve the Poisson equation on the 2D domain $-1 \le x,y \le 1$

  $$-\frac{\partial^2 u}{\partial x^2} -\frac{\partial^2 u}{\partial x^2} = 1$$ 
  $$u(-1, y) = 0 \qquad u( 1, y) = 0$$ 
  $$u( x,-1) = 0 \qquad u( x, 1) = 0$$ 

  Compare against the analytical solution given by

  \begin{align}
    u(x,y) = \frac{1-x^2}{2} - \frac{16}{\pi^3} \sum_{\substack{k=0 \\ k\text{
      odd}}}^\infty \biggl\{ 
      &\frac{\sin(k\pi(1+x)/2)}{k^3\sinh(k\pi)} \\
      &\times (\sinh(k\pi(1+y)/2) + \sinh(k\pi(1-y)/2)) 
      \biggr\}
  \end{align}

\item  For each of the problems listed above you had to solve a linear system of
  equations. Scipy has solvers both for dense matrices
    (\href{https://docs.scipy.org/doc/scipy/reference/generated/scipy.linalg.solve.html}{scipy.linalg.solve})
    and for sparse representations
    (\href{https://docs.scipy.org/doc/scipy/reference/generated/scipy.sparse.linalg.spsolve.html#scipy.sparse.linalg.spsolve}{scipy.sparse.linalg.spsolve}).
    Pick one of the problems above and time how long these solves take with varying
    number of discrete points $N$. How does the compute time scale with $N$? Explain the
observed scaling for the two matrix representations?  \end{enumerate}

\subsection*{Exercise 2 - time integration}

\begin{enumerate}[label=\alph*.]
  \item Solve for $u(x, t)$ the 1D Fisher equation given below over $0 \le t \le 42$ and $0 \le x \le
  20$, using an explicit Euler time-stepping
  scheme. Measure the speed of the expected travelling wave solution and ensure that it
    is equal to $c=2$.

    $$ \frac{du}{dt} = \frac{\partial^2 u}{\partial x^2} + u (1-u)$$
    $$ u(x,0) = \begin{cases} 
                    1 & \text{for } x \le 3 \\ 
                    0 & \text{otherwise}
                \end{cases}
    $$
    $$u(0, t) = 1$$
    $$u(42, t) = 0$$


  \item Solve for $u(x,t)$ the biharmonic equation given below over $0 \le t \le 0.5$ and
    $-1 \le x \le 1$, using both an explicit and implicit Euler time-stepping scheme.
    Comment on the differences in the required time-step size

    $$\frac{du}{dt} = \frac{\partial^4 u}{\partial x^4}$$
    $$ u(x,0) = \begin{cases} 
                    1 & \text{for } |x| \le 3 \\ 
                    0 & \text{otherwise}
                \end{cases}
    $$
    $$u(-1, t) = 0$$
    $$u(1, t) = 0$$

  \item Solve for $u(x,t)$ the Kuramoto Sivashinsky equation given below over $0 \le t
    \le 0.5$ and $-1 \le x \le 1$. Use the Euler method to discretise the time derivative,
    but treat the non-linear term explictly, and the rest of the terms implicitly. Why is it
    preferrable to treat the non-linear term explicitly in this case? You should obtain
    an oscillating, chaotic behaviour. Longer waves grow because of the
    $-\frac{\partial^2 u}{\partial x^2}$ term, shorter waves decay because of the
    $-\frac{\partial^4 u}{\partial x^4}$ term, and the non-linear term transfers energy
    from long to short waves.

    $$\frac{du}{dt} = -\frac{\partial^2 u}{\partial x^2} - \frac{\partial^4 u}{\partial
    x^4} - \frac{\partial}{\partial x} (u^2/2)$$
    $$ u(x,0) = \begin{cases} 
                    1 & \text{for } |x| \le 3 \\ 
                    0 & \text{otherwise}
                \end{cases}
    $$
    $$u(-1, t) = 0$$
    $$u(1, t) = 0$$
\end{enumerate}

 
\end{document}
